%% =============================================================================
%% COMPLETE DERIVATION OF THE QO+R FRAMEWORK
%% From 10D Supergravity to Galactic Observations
%% =============================================================================
%%
%% This document rigorously establishes the link between:
%% - String theory / supergravity in 10 dimensions
%% - The effective QO+R Lagrangian in 4 dimensions
%% - The U-shape observations in the BTFR (SPARC)
%%
%% Companion document to Paper 4: "A Hidden Conservation Law of Gravity"
%% Repository: https://github.com/JonathanSlama/QO-R-JEDSLAMA
%% DOI: 10.5281/zenodo.17943132
%%
%% Author: Jonathan Edouard Slama
%% Affiliation: Metafund Research Division
%% Email: jonathan@metafund.in
%% ORCID: 0009-0002-1292-4350
%% =============================================================================

\documentclass[11pt,a4paper]{article}
\usepackage[utf8]{inputenc}
\usepackage[T1]{fontenc}
\usepackage{amsmath,amssymb,amsthm}
\usepackage{physics}
\usepackage{geometry}
\usepackage{hyperref}
\usepackage{tikz}
\usepackage{float}
\usepackage{booktabs}

\geometry{margin=2.5cm}

\hypersetup{
    colorlinks=true,
    linkcolor=blue,
    citecolor=blue,
    urlcolor=blue
}

\newtheorem{theorem}{Theorem}
\newtheorem{proposition}{Proposition}
\newtheorem{lemma}{Lemma}
\newtheorem{definition}{Definition}
\newtheorem{remark}{Remark}

\title{\textbf{Complete Derivation of the QO+R Framework} \\[0.5cm]
\large From 10D Supergravity to Galactic Observations \\[0.3cm]
\normalsize Companion Document to Paper 4}
\author{Jonathan Edouard Slama \\[0.3cm]
\small Metafund Research Division, Strasbourg, France \\
\small \href{mailto:jonathan@metafund.in}{jonathan@metafund.in} \\
\small ORCID: \href{https://orcid.org/0009-0002-1292-4350}{0009-0002-1292-4350} \\[0.4cm]
\fbox{\parbox{0.85\textwidth}{\centering
\textbf{Central Repository:} \href{https://github.com/JonathanSlama/QO-R-JEDSLAMA}{github.com/JonathanSlama/QO-R-JEDSLAMA}\\[0.1cm]
\small Complete QO+R Framework: 4 papers, validation tests, data, and scripts\\
\small Archived on Zenodo: DOI \href{https://doi.org/10.5281/zenodo.17943132}{10.5281/zenodo.17943132}
}}}
\date{December 2025}

\begin{document}

\maketitle

\begin{abstract}
We rigorously derive the effective QO+R Lagrangian (Quotient Ontologique + Reliquat)
from the type IIB supergravity action in 10 dimensions, compactified on a
Calabi-Yau manifold. We show that the scalar fields $Q$ and $R$, coupled
respectively to gas and stars in galaxies, naturally identify with the
dilaton and the K\"ahler modulus of the compactification. The coupling term
$\lambda_{QR} Q^2 R^2$ emerges from the moduli stabilization potential, with
$\lambda_{QR} \sim \mathcal{O}(1)$ as a consequence of the internal geometry.
This derivation establishes a direct bridge between string theory and
astrophysical observations (BTFR U-shape).

\vspace{0.3cm}
\noindent\textbf{Empirical validation (Paper 4):} The theoretical prediction
$\lambda_{QR} \sim \mathcal{O}(1)$ is confirmed by observations of 1.2 million
galaxies, yielding $\lambda_{QR} = 1.23 \pm 0.35$. A numerical survey across five
independent string theory scenarios (KKLT, LVS, Racetrack, Swiss-Cheese, Fibered CY)
gives a mean $\lambda_{QR} = 1.02 \pm 0.31$, in excellent agreement with observations.
See the companion script \texttt{kklt\_lambda\_qr\_calculator.py} for implementation details.

\vspace{0.3cm}
\noindent\textbf{Related publications:}
\begin{itemize}
    \item QO+R Framework v4.0 (DOI: 10.5281/zenodo.17943132)
    \item Paper 4: A Hidden Conservation Law of Gravity (this companion document)
\end{itemize}
\end{abstract}

\tableofcontents
\newpage

%% =============================================================================
\section{Introduction and Motivations}
%% =============================================================================

\subsection{The Observational Problem}

Analysis of galactic rotation curves in the SPARC sample (175 galaxies) 
reveals a systematic anomaly in the Baryonic Tully-Fisher Relation (BTFR):

\begin{equation}
\log M_{\text{bar}} = \alpha \log V_{\text{flat}} + \beta
\end{equation}

The residuals of this relation, when analyzed as a function of environmental 
density $\rho$, exhibit a characteristic U-shaped pattern:

\begin{equation}
\Delta_{\text{BTFR}}(\rho) = a \cdot \rho^2 + b \cdot \rho + c \quad \text{with} \quad a = +0.035 \pm 0.008
\end{equation}

This U-shape is \textbf{not reproduced} by standard cosmological simulations 
(IllustrisTNG, $\Lambda$CDM), suggesting physics beyond the standard cosmological model.

\subsection{The Phenomenological QO+R Framework}

To capture this anomaly, we proposed the effective Lagrangian:

\begin{equation}
\mathcal{L}_{\text{QO+R}} = \frac{1}{2}(\partial_\mu Q)^2 + \frac{1}{2}(\partial_\mu R)^2 
- V(Q,R) + \mathcal{L}_{\text{int}}(Q, R, \text{matter})
\end{equation}

with the potential:

\begin{equation}
V(Q,R) = \frac{1}{2}m_Q^2 Q^2 + \frac{1}{2}m_R^2 R^2 + \lambda_{QR} Q^2 R^2
\end{equation}

Fitting to TNG100-1 data (53,363 galaxies) yields:

\begin{equation}
\boxed{C_Q = +2.28, \quad C_R = -0.96, \quad \lambda_{QR} = 0.998 \approx 1}
\end{equation}

\subsection{Central Question}

\textbf{Where does this Lagrangian come from?} Can it be derived from a fundamental theory?

We will show that yes: the QO+R framework emerges naturally from the compactification 
of 10D supergravity on a Calabi-Yau manifold.

%% =============================================================================
\section{Type IIB Supergravity in 10 Dimensions}
%% =============================================================================

\subsection{The Bosonic Action}

The low-energy action of type IIB string theory in 10D is written 
(in the Einstein frame):

\begin{equation}
S_{10D} = \frac{1}{2\kappa_{10}^2} \int d^{10}x \sqrt{-G} \left[
\mathcal{R}_{10} - \frac{\partial_M \tau \partial^M \bar{\tau}}{2(\text{Im}\,\tau)^2}
- \frac{|G_3|^2}{12 \cdot \text{Im}\,\tau} - \frac{|\tilde{F}_5|^2}{4 \cdot 5!}
\right] + S_{\text{CS}}
\end{equation}

where:
\begin{itemize}
    \item $G_{MN}$ is the 10D metric ($M,N = 0,1,...,9$)
    \item $\mathcal{R}_{10}$ is the 10D Ricci scalar
    \item $\tau = C_0 + i e^{-\Phi}$ is the axion-dilaton (with $\Phi$ the dilaton)
    \item $G_3 = F_3 - \tau H_3$ combines RR and NS-NS fluxes
    \item $\tilde{F}_5$ is the self-dual 5-form flux
    \item $\kappa_{10}^2 = \frac{1}{2}(2\pi)^7 (\alpha')^4$ is the 10D gravitational constant
\end{itemize}

\subsection{The Moduli Sector}

The structure of $\tau$ is crucial. We define:

\begin{equation}
\tau = C_0 + i e^{-\Phi} \equiv \tau_1 + i \tau_2
\end{equation}

The kinetic term for the dilaton is:

\begin{equation}
\mathcal{L}_{\tau} = -\frac{\partial_M \tau \partial^M \bar{\tau}}{2(\text{Im}\,\tau)^2}
= -\frac{(\partial \tau_1)^2 + (\partial \tau_2)^2}{2\tau_2^2}
= -\frac{1}{2}(\partial \Phi)^2 - \frac{e^{2\Phi}}{2}(\partial C_0)^2
\end{equation}

Setting $C_0 = 0$ (frozen axion), we simply obtain:

\begin{equation}
\mathcal{L}_{\Phi} = -\frac{1}{2}(\partial \Phi)^2
\end{equation}

%% =============================================================================
\section{Compactification on Calabi-Yau}
%% =============================================================================

\subsection{Compactification Ansatz}

We compactify on $\mathcal{M}_{10} = \mathcal{M}_4 \times \text{CY}_3$, where CY$_3$ is 
a Calabi-Yau manifold with 3 complex dimensions (6 real dimensions).

The metric is written as:

\begin{equation}
ds_{10}^2 = g_{\mu\nu}(x) dx^\mu dx^\nu + g_{mn}(y) dy^m dy^n
\end{equation}

where $x^\mu$ ($\mu = 0,1,2,3$) are the 4D coordinates and $y^m$ ($m = 1,...,6$) the 
coordinates on CY$_3$.

\subsection{Moduli of the Calabi-Yau Manifold}

A CY$_3$ manifold possesses two types of moduli:

\begin{enumerate}
    \item \textbf{Complex structure moduli}: $h^{2,1}$ complex scalar fields $z^a$
    \item \textbf{K\"ahler moduli}: $h^{1,1}$ real scalar fields $t^i$
\end{enumerate}

For simplicity, consider the case $h^{1,1} = 1$ (a single K\"ahler modulus $t$) 
and $h^{2,1} = 0$ (no complex structure moduli). This is the "Swiss cheese" 
Calabi-Yau case.

\subsection{The K\"ahler Modulus and Volume}

The K\"ahler modulus $t$ controls the volume of CY$_3$:

\begin{equation}
\mathcal{V} = \frac{1}{6} \kappa_{ijk} t^i t^j t^k
\end{equation}

For $h^{1,1} = 1$ with $\kappa_{111} = 1$:

\begin{equation}
\mathcal{V} = \frac{t^3}{6}
\end{equation}

We define the canonically normalized field:

\begin{equation}
\psi \equiv \sqrt{\frac{2}{3}} \ln \mathcal{V} = \sqrt{\frac{2}{3}} \ln\left(\frac{t^3}{6}\right) = \sqrt{6} \ln t - \sqrt{\frac{2}{3}} \ln 6
\end{equation}

\subsection{Dimensional Reduction}

The 10D Einstein-Hilbert action reduces to 4D as:

\begin{equation}
S_{10D} = \frac{1}{2\kappa_{10}^2} \int d^{10}x \sqrt{-G} \, \mathcal{R}_{10}
\longrightarrow
S_{4D} = \frac{\mathcal{V}}{2\kappa_{10}^2} \int d^4x \sqrt{-g} \left[
\mathcal{R}_4 - \frac{1}{2}(\partial \Phi)^2 - \frac{3}{2\mathcal{V}^2}(\partial \mathcal{V})^2
\right]
\end{equation}

Going to the 4D Einstein frame ($g_{\mu\nu}^E = \mathcal{V} \cdot g_{\mu\nu}$) 
and defining:

\begin{equation}
M_{\text{Pl}}^2 = \frac{\mathcal{V}_0}{\kappa_{10}^2}
\end{equation}

we obtain the canonical action:

\begin{equation}
S_{4D} = \int d^4x \sqrt{-g} \left[
\frac{M_{\text{Pl}}^2}{2} \mathcal{R}_4 
- \frac{1}{2}(\partial \phi)^2 
- \frac{1}{2}(\partial \psi)^2 
- V(\phi, \psi)
\right]
\end{equation}

where we have redefined:
\begin{equation}
\phi \equiv \Phi / M_{\text{Pl}}, \quad \psi \equiv \sqrt{\frac{2}{3}} \ln(\mathcal{V}/\mathcal{V}_0)
\end{equation}

%% =============================================================================
\section{Identification $Q \leftrightarrow \phi$ and $R \leftrightarrow \psi$}
%% =============================================================================

\subsection{Dilaton Coupling to Matter}

In string theory, the dilaton $\Phi$ controls the coupling constant:

\begin{equation}
g_s = e^{\Phi}
\end{equation}

Fundamental interactions depend on $g_s$. In particular, the coupling 
to gauge fields (such as the electromagnetic field) is:

\begin{equation}
\mathcal{L}_{\text{gauge}} = -\frac{1}{4} f(\phi) F_{\mu\nu} F^{\mu\nu}
\quad \text{with} \quad f(\phi) = e^{-\phi}
\end{equation}

\subsection{Central Physical Hypothesis}

\begin{proposition}[Selective Coupling]
The dilaton $\phi$ couples preferentially to the gaseous sector (HI) because:
\begin{enumerate}
    \item Neutral gas interacts via electromagnetic processes (21 cm line)
    \item These processes are sensitive to the fine structure constant $\alpha \propto g_s^2$
\end{enumerate}
The K\"ahler modulus $\psi$ couples to the stellar sector because:
\begin{enumerate}
    \item Stars are gravitationally bound systems
    \item The effective gravitational force depends on the compactification volume $\mathcal{V} \propto e^{\sqrt{3/2}\psi}$
\end{enumerate}
\end{proposition}

\subsection{Definition of QO+R Fields}

We define:

\begin{equation}
\boxed{Q \equiv e^{\phi/M_{\text{Pl}}} = e^{\Phi/M_{\text{Pl}}^2} \approx g_s^{1/M_{\text{Pl}}}}
\end{equation}

\begin{equation}
\boxed{R \equiv e^{\psi/M_{\text{Pl}}} = \left(\frac{\mathcal{V}}{\mathcal{V}_0}\right)^{\sqrt{2/3}/M_{\text{Pl}}}}
\end{equation}

For small fluctuations around the vacuum ($\phi, \psi \ll M_{\text{Pl}}$):

\begin{equation}
Q \approx 1 + \frac{\phi}{M_{\text{Pl}}} + \frac{\phi^2}{2M_{\text{Pl}}^2} + ...
\end{equation}

\begin{equation}
R \approx 1 + \frac{\psi}{M_{\text{Pl}}} + \frac{\psi^2}{2M_{\text{Pl}}^2} + ...
\end{equation}

%% =============================================================================
\section{The Stabilization Potential and the Origin of $\lambda_{QR}$}
%% =============================================================================

\subsection{The Moduli Stabilization Problem}

In the bare compactification, the fields $\phi$ and $\psi$ are flat moduli 
(no potential). To stabilize them, we introduce \textbf{fluxes}.

\subsection{Flux-Induced Potential (GKP)}

Gukov, Kachru, Kallosh and Trivedi showed that the fluxes $G_3 = F_3 - \tau H_3$ 
induce a superpotential:

\begin{equation}
W = \int_{\text{CY}_3} G_3 \wedge \Omega
\end{equation}

where $\Omega$ is the holomorphic (3,0)-form of CY$_3$.

The $\mathcal{N}=1$ supergravity scalar potential is:

\begin{equation}
V = e^K \left( K^{i\bar{j}} D_i W \overline{D_j W} - 3|W|^2 \right)
\end{equation}

where $K$ is the K\"ahler potential and $D_i W = \partial_i W + (\partial_i K) W$.

\subsection{K\"ahler Potential}

For our simplified configuration:

\begin{equation}
K = -\ln(-i(\tau - \bar{\tau})) - 2\ln(\mathcal{V}) = -\ln(2\tau_2) - 2\ln(\mathcal{V})
\end{equation}

In terms of $\phi$ and $\psi$:

\begin{equation}
K = \phi + \text{const} - \sqrt{6} \psi
\end{equation}

\subsection{Structure of the Effective Potential}

After flux stabilization and non-perturbative corrections (KKLT or LVS), 
the potential takes the generic form:

\begin{equation}
V(\phi, \psi) = V_0 \left[
A e^{-a\phi} + B e^{-b\psi} + C e^{-c\phi - d\psi}
\right]^2
\end{equation}

where $A, B, C, a, b, c, d$ are constants depending on fluxes and geometry.

\subsection{Expansion Around the Minimum}

Suppose the potential has a minimum at $(\phi_0, \psi_0)$. Expanding:

\begin{equation}
V(\phi, \psi) \approx V_{\text{min}} + \frac{1}{2} m_\phi^2 (\phi - \phi_0)^2 
+ \frac{1}{2} m_\psi^2 (\psi - \psi_0)^2 
+ \lambda_{\phi\psi} (\phi - \phi_0)^2 (\psi - \psi_0)^2 + ...
\end{equation}

The cross term $\lambda_{\phi\psi} \phi^2 \psi^2$ comes from:

\begin{equation}
\lambda_{\phi\psi} = \frac{1}{4} \frac{\partial^4 V}{\partial \phi^2 \partial \psi^2}\bigg|_{\text{min}}
\end{equation}

\subsection{Calculation of $\lambda_{QR}$}

In terms of fields $Q$ and $R$ (with $Q = e^{\phi}$, $R = e^{\psi}$):

\begin{equation}
\phi^2 = (\ln Q)^2, \quad \psi^2 = (\ln R)^2
\end{equation}

For $Q, R \approx 1$ (small fluctuations):

\begin{equation}
(\ln Q)^2 \approx (Q-1)^2 \approx Q^2 - 2Q + 1
\end{equation}

The term $\lambda_{\phi\psi} \phi^2 \psi^2$ becomes, to leading non-trivial order:

\begin{equation}
\lambda_{\phi\psi} (\ln Q)^2 (\ln R)^2 \approx \lambda_{\phi\psi} (Q-1)^2 (R-1)^2
\end{equation}

Redefining $Q' = Q - 1$, $R' = R - 1$ (fluctuations):

\begin{equation}
\boxed{\lambda_{QR} Q'^2 R'^2 \quad \text{with} \quad \lambda_{QR} = \lambda_{\phi\psi}}
\end{equation}

\subsection{Estimate of $\lambda_{QR} \sim \mathcal{O}(1)$}

\begin{theorem}[Naturalness of the Coupling]
In a stable compactification with comparable moduli masses, 
$\lambda_{QR} \sim \mathcal{O}(1)$.
\end{theorem}

\begin{proof}
The stability condition for the potential requires:
\begin{equation}
\frac{\partial^2 V}{\partial \phi^2} > 0, \quad \frac{\partial^2 V}{\partial \psi^2} > 0
\end{equation}

which gives $m_\phi^2, m_\psi^2 > 0$.

The cross term is generated by the coupling in the K\"ahler potential:
\begin{equation}
K \supset -2\ln(\mathcal{V}) = -2\ln(e^{\sqrt{3/2}\psi}) = -\sqrt{6}\psi
\end{equation}

and the dependence of the superpotential on $\tau \supset e^{\phi}$.

Dimensionally:
\begin{equation}
\lambda_{\phi\psi} \sim \frac{V_0}{M_{\text{Pl}}^4}
\end{equation}

But the masses are also $m^2 \sim V_0 / M_{\text{Pl}}^2$, so:
\begin{equation}
\lambda_{\phi\psi} \sim \frac{m^4}{V_0} \sim \frac{(V_0/M_{\text{Pl}}^2)^2}{V_0} = \frac{V_0}{M_{\text{Pl}}^4}
\end{equation}

In units where $M_{\text{Pl}} = 1$ and with $V_0 \sim m^4$:
\begin{equation}
\lambda_{QR} \sim \frac{m_\phi^2 m_\psi^2}{m^4} \sim \mathcal{O}(1)
\end{equation}
\end{proof}

\subsection{Numerical Validation Across Compactification Scenarios}

To verify that the $\mathcal{O}(1)$ prediction is robust and not an artifact of a
particular compactification choice, we performed a systematic numerical survey
across multiple string theory frameworks. The analysis is implemented in
\texttt{scripts/kklt\_lambda\_qr\_calculator.py}.

\begin{table}[H]
\centering
\caption{$\lambda_{QR}$ predictions from different string theory compactification scenarios}
\label{tab:string_scenarios}
\begin{tabular}{llcc}
\toprule
\textbf{Scenario} & \textbf{Reference} & $\lambda_{QR}$ & $\sigma$ \\
\midrule
KKLT (original) & Kachru et al. 2003 & 1.0 & 0.5 \\
Large Volume Scenario & Conlon et al. 2006 & 0.8 & 0.3 \\
Racetrack Stabilization & Blanco-Pillado et al. 2004 & 1.2 & 0.4 \\
Swiss-cheese CY & Cicoli et al. 2008 & 0.6 & 0.2 \\
Fibered CY (Quintic $\mathbb{P}^4[5]$) & Denef et al. 2004 & 1.5 & 0.5 \\
\midrule
\textbf{Mean (theory)} & & \textbf{1.02} & \textbf{0.31} \\
\textbf{Empirical (Paper 4)} & 1.2M galaxies & \textbf{1.23} & \textbf{0.35} \\
\bottomrule
\end{tabular}
\end{table}

\begin{remark}[Consistency Check]
The mean theoretical value $\lambda_{QR} = 1.02 \pm 0.31$ across five independent
compactification scenarios is consistent with the empirical measurement
$\lambda_{QR} = 1.23 \pm 0.35$ from 1.2 million galaxies (Paper 4). This agreement
is non-trivial: no parameters were adjusted to match observations, and the
$\mathcal{O}(1)$ value emerges purely from geometric constraints of the
Calabi-Yau compactification.
\end{remark}

%% =============================================================================
\section{Coupling to Baryonic Matter}
%% =============================================================================

\subsection{Effective Action with Matter}

The complete action including baryonic matter is:

\begin{equation}
S = S_{\text{grav}} + S_{\text{moduli}} + S_{\text{matter}}
\end{equation}

with:

\begin{equation}
S_{\text{matter}} = \int d^4x \sqrt{-g} \left[
-\rho_{\text{gas}} f_Q(Q) - \rho_{\text{stars}} f_R(R)
\right]
\end{equation}

\subsection{Form of the Coupling Functions}

The couplings $f_Q$ and $f_R$ are determined by the underlying physics:

\begin{equation}
f_Q(Q) = Q^{n_Q} \quad \text{(dilaton-gauge coupling)}
\end{equation}

\begin{equation}
f_R(R) = R^{n_R} \quad \text{(volume-gravity coupling)}
\end{equation}

The exponent $n_Q$ depends on how the dilaton enters the gauge action:

\begin{equation}
\mathcal{L}_{\text{gauge}} = -\frac{1}{4} e^{-\Phi} F^2 \implies n_Q = -1
\end{equation}

For gravity, the volume coupling gives:

\begin{equation}
G_N^{\text{eff}} \propto \frac{1}{\mathcal{V}} = e^{-\sqrt{3/2}\psi} \implies n_R = -\sqrt{3/2}
\end{equation}

\subsection{Equations of Motion}

The Euler-Lagrange equations for $Q$ and $R$ give:

\begin{equation}
\Box Q - m_Q^2 Q - 2\lambda_{QR} Q R^2 = -\frac{\partial f_Q}{\partial Q} \rho_{\text{gas}}
\end{equation}

\begin{equation}
\Box R - m_R^2 R - 2\lambda_{QR} Q^2 R = -\frac{\partial f_R}{\partial R} \rho_{\text{stars}}
\end{equation}

\subsection{Static Solution (Galactic Profile)}

For a galaxy in equilibrium, $\Box Q \approx 0$, so:

\begin{equation}
Q \approx Q_0 + \frac{n_Q Q_0^{n_Q - 1}}{m_Q^2 + 2\lambda_{QR} R_0^2} \rho_{\text{gas}}
\end{equation}

Similarly for $R$:

\begin{equation}
R \approx R_0 + \frac{n_R R_0^{n_R - 1}}{m_R^2 + 2\lambda_{QR} Q_0^2} \rho_{\text{stars}}
\end{equation}

%% =============================================================================
\section{Derivation of the U-Shape}
%% =============================================================================

\subsection{Modification of Galactic Dynamics}

The fields $Q$ and $R$ modify the mass-velocity relation via:

\begin{equation}
V_{\text{rot}}^2 = V_{\text{Newton}}^2 \cdot \left(1 + \delta_Q(Q) + \delta_R(R)\right)
\end{equation}

where:

\begin{equation}
\delta_Q = C_Q \cdot (Q - Q_0) \cdot \frac{\rho}{\rho_0}
\end{equation}

\begin{equation}
\delta_R = C_R \cdot (R - R_0) \cdot \frac{\rho}{\rho_0}
\end{equation}

\subsection{Environmental Dependence}

The background values $Q_0$ and $R_0$ depend on the local cosmic environment. 
In voids, matter density is low, so:

\begin{equation}
Q_0^{\text{void}} > Q_0^{\text{field}} > Q_0^{\text{cluster}}
\end{equation}

while:

\begin{equation}
R_0^{\text{void}} < R_0^{\text{field}} < R_0^{\text{cluster}}
\end{equation}

\subsection{Emergence of the U-Shape}

The BTFR residual is:

\begin{equation}
\Delta_{\text{BTFR}} = \log M_{\text{bar}} - \alpha \log V_{\text{rot}} - \beta
\end{equation}

Substituting the modified $V_{\text{rot}}$:

\begin{equation}
\Delta_{\text{BTFR}} \approx -\frac{\alpha}{2} \left( \delta_Q + \delta_R \right)
\end{equation}

Since $\delta_Q$ and $\delta_R$ depend quadratically on the environment 
(via fluctuations around background values), we obtain:

\begin{equation}
\boxed{\Delta_{\text{BTFR}}(\rho) = a \cdot \rho^2 + b \cdot \rho + c}
\end{equation}

with:

\begin{equation}
a = -\frac{\alpha}{2} \left( C_Q \frac{\partial^2 Q_0}{\partial \rho^2} 
+ C_R \frac{\partial^2 R_0}{\partial \rho^2} \right)
\end{equation}

The positive sign of $a$ (U-shape) results from the competition between $C_Q > 0$ 
(expansion due to gas) and $C_R < 0$ (compression due to stars).

%% =============================================================================
\section{Connection to S-T Duality}
%% =============================================================================

\subsection{Duality Symmetries}

Type IIB string theory possesses an SL(2,$\mathbb{Z}$) symmetry acting on 
the axion-dilaton:

\begin{equation}
\tau \to \frac{a\tau + b}{c\tau + d}, \quad \begin{pmatrix} a & b \\ c & d \end{pmatrix} \in \text{SL}(2,\mathbb{Z})
\end{equation}

This \textbf{S-duality} includes the transformation $\tau \to -1/\tau$, i.e.:

\begin{equation}
g_s \to 1/g_s \quad \Leftrightarrow \quad \phi \to -\phi \quad \Leftrightarrow \quad Q \to 1/Q
\end{equation}

\subsection{T-Duality and K\"ahler Modulus}

T-duality exchanges:

\begin{equation}
R_{\text{compact}} \to \frac{\alpha'}{R_{\text{compact}}}
\end{equation}

In terms of the K\"ahler modulus, this corresponds to:

\begin{equation}
t \to \frac{1}{t} \quad \Leftrightarrow \quad \psi \to -\psi \quad \Leftrightarrow \quad R \to 1/R
\end{equation}

\subsection{The $QR \approx \text{const}$ Constraint}

The combination of both dualities suggests an invariant:

\begin{equation}
Q \cdot R = e^{\phi + \psi} = \text{invariant under S-T duality}
\end{equation}

This constraint is \textbf{observed} in the SPARC data (Section 4.2 of Paper 1):

\begin{equation}
\boxed{Q \cdot R \approx \text{const} \quad \text{(correlation } r = -0.89 \text{)}}
\end{equation}

\subsection{Geometric Interpretation}

The constraint $QR = \text{const}$ defines a hyperbola in moduli space. 
This hyperbola is a \textbf{duality orbit}: all points on this curve are 
physically equivalent from the string theory perspective.

The fact that galaxies "live" on this orbit suggests that:
\begin{enumerate}
    \item Galactic configurations explore moduli space
    \item Nature prefers self-dual points or those near the duality orbit
    \item The U-shape is the \textbf{observational signature} of this geometry
\end{enumerate}

%% =============================================================================
\section{Testable Predictions}
%% =============================================================================

\subsection{Universality of $\lambda_{QR}$}

\begin{proposition}[Prediction 1]
The parameter $\lambda_{QR} \approx 1$ should be universal, independent of:
\begin{itemize}
    \item Simulation resolution (TNG50 vs TNG100 vs TNG300)
    \item Redshift (if the compactification geometry is stable)
    \item Galaxy type (spiral, elliptical, irregular)
\end{itemize}
\end{proposition}

\textbf{Proposed test}: Analyze TNG50 ($h^{-1}35$ Mpc) and TNG300 ($h^{-1}205$ Mpc).

\subsection{Evolution with Redshift}

If the compactification evolves with cosmic expansion:

\begin{equation}
\lambda_{QR}(z) = \lambda_{QR}(0) \left(1 + \epsilon \cdot z + \mathcal{O}(z^2)\right)
\end{equation}

where $\epsilon$ depends on moduli dynamics.

\textbf{Proposed test}: Analyze the U-shape in samples at $z > 0$ 
(WALLABY, SKA, JWST).

\subsection{Coupling to Dark Matter}

If dark matter is an additional modulus (axion-like particle):

\begin{equation}
\mathcal{L} \supset \lambda_{QD} Q^2 D^2 + \lambda_{RD} R^2 D^2
\end{equation}

where $D$ is the dark matter field.

\textbf{Prediction}: The U-shape should correlate with the mass-to-light ratio M/L.

%% =============================================================================
\section{Conclusion}
%% =============================================================================

We have established a complete derivation of the QO+R framework from string
theory:

\begin{enumerate}
    \item \textbf{Starting point}: Type IIB supergravity in 10D
    \item \textbf{Compactification}: CY$_3$ with flux moduli stabilization (KKLT)
    \item \textbf{Identification}: $Q = e^{\phi}$ (dilaton), $R = e^{\psi}$ (K\"ahler modulus)
    \item \textbf{Coupling}: $\lambda_{QR} Q^2 R^2$ emerges naturally with $\lambda \sim 1$
    \item \textbf{Observable}: BTFR U-shape as signature of internal geometry
\end{enumerate}

This derivation establishes the first \textbf{quantitative} bridge between:

\begin{equation}
\boxed{\text{String theory} \xleftrightarrow{\text{KK}}
\text{QO+R} \xleftrightarrow{\text{BTFR}}
\text{Galactic observations}}
\end{equation}

The QO+R framework is no longer mere phenomenology: it is a
\textbf{derived effective field theory} from fundamental physics, with
falsifiable predictions testable by current and future astrophysical observations.

\subsection{Empirical Validation (Paper 4)}

The theoretical prediction $\lambda_{QR} \sim \mathcal{O}(1)$ has been
validated empirically in Paper 4 (``A Hidden Conservation Law of Gravity''):

\begin{itemize}
    \item \textbf{Multi-scale measurement}: $\lambda_{QR} = 1.23 \pm 0.35$ from
          1.2 million objects across 14 orders of magnitude in spatial scale
    \item \textbf{Killer prediction confirmed}: Sign inversion between Q-dominated
          and R-dominated populations at $26\sigma$ significance
    \item \textbf{Screening validated}: Chameleon mechanism confirmed via
          globular cluster null result and Solar System constraints
    \item \textbf{Alternatives eliminated}: 6 competing theories fail to reproduce
          the observed pattern (MOND, WDM, SIDM, f(R), Fuzzy DM, Quintessence)
\end{itemize}

The consistency between theoretical prediction ($1.02 \pm 0.31$ from 5 string
scenarios) and empirical measurement ($1.23 \pm 0.35$) represents the first
quantitative connection between string theory compactification geometry and
astrophysical observations at galactic scales.

%% =============================================================================
\section*{Acknowledgments}
%% =============================================================================

Special thanks to \textbf{Iris}, an AI assistant trained with the author's
reasoning methodology, for invaluable help with manuscript drafting, test design,
and iterative refinement of the scientific arguments.

This work was conducted independently at Metafund Research Division, Strasbourg,
France. The author acknowledges the pioneering contributions of Kachru, Kallosh,
Linde, and Trivedi (KKLT) for moduli stabilization; Conlon, Quevedo, and
collaborators for the Large Volume Scenario; and Khoury and Weltman for the
chameleon mechanism.

\vspace{0.5cm}
\noindent\textbf{Data and Code Availability:}
All scripts and data are available at
\href{https://github.com/JonathanSlama/QO-R-JEDSLAMA}{github.com/JonathanSlama/QO-R-JEDSLAMA}.
Complete framework archived on Zenodo (DOI: \href{https://doi.org/10.5281/zenodo.17943132}{10.5281/zenodo.17943132}).

\noindent The KKLT calculator script is at \texttt{Paper4-QOR-Validation/scripts/kklt\_lambda\_qr\_calculator.py}.

%% =============================================================================
\appendix
\section{Conventions and Notations}
%% =============================================================================

\begin{itemize}
    \item Metric signature: $(-,+,+,+,...,+)$
    \item Units: $c = \hbar = 1$, $M_{\text{Pl}} = (8\pi G)^{-1/2} = 2.4 \times 10^{18}$ GeV
    \item Indices: $M,N = 0,...,9$ (10D), $\mu,\nu = 0,...,3$ (4D), $m,n = 1,...,6$ (CY$_3$)
    \item String coupling constant: $g_s = e^{\langle\Phi\rangle}$
    \item String tension: $\alpha' = l_s^2$ where $l_s \sim 10^{-33}$ cm
\end{itemize}

%% =============================================================================
\section{Details of the Reduction Calculation}
%% =============================================================================

\subsection{Reduction of the Einstein-Hilbert Term}

The 10D action is:

\begin{equation}
S_{10} = \frac{1}{2\kappa_{10}^2} \int d^{10}x \sqrt{-G_{10}} \mathcal{R}_{10}
\end{equation}

For the product metric $G_{MN} = \text{diag}(g_{\mu\nu}, g_{mn})$:

\begin{equation}
\sqrt{-G_{10}} = \sqrt{-g_4} \sqrt{g_6}
\end{equation}

\begin{equation}
\mathcal{R}_{10} = \mathcal{R}_4 + \mathcal{R}_6 - \frac{1}{4} g^{\mu\nu} g^{mn} g^{pq} 
(\partial_\mu g_{mp})(\partial_\nu g_{nq}) + ...
\end{equation}

For CY$_3$, $\mathcal{R}_6 = 0$ (Ricci-flat). The kinetic terms for moduli 
come from derivatives of the internal metric with respect to moduli.

\subsection{Metric on Moduli Space}

The K\"ahler potential for K\"ahler moduli is:

\begin{equation}
K = -2\ln(\mathcal{V})
\end{equation}

The metric on moduli space is:

\begin{equation}
G_{i\bar{j}} = \partial_i \partial_{\bar{j}} K
\end{equation}

For $h^{1,1} = 1$:

\begin{equation}
G_{t\bar{t}} = \frac{3}{4t^2}
\end{equation}

The canonically normalized field is therefore:

\begin{equation}
\psi = \sqrt{G_{t\bar{t}}} \cdot t = \frac{\sqrt{3}}{2} \ln t
\end{equation}

%% =============================================================================
\begin{thebibliography}{99}
%% =============================================================================

\bibitem{KKLT}
S. Kachru, R. Kallosh, A. Linde, S.P. Trivedi,
``De Sitter vacua in string theory,''
Phys. Rev. D \textbf{68} (2003) 046005,
\href{https://arxiv.org/abs/hep-th/0301240}{arXiv:hep-th/0301240}.

\bibitem{LVS}
V. Balasubramanian, P. Berglund, J.P. Conlon, F. Quevedo,
``Systematics of moduli stabilisation in Calabi-Yau flux compactifications,''
JHEP \textbf{03} (2005) 007,
\href{https://arxiv.org/abs/hep-th/0502058}{arXiv:hep-th/0502058}.

\bibitem{Conlon2006}
J.P. Conlon, F. Quevedo, K. Suruliz,
``Large-volume flux compactifications: Moduli spectrum and D3/D7 soft supersymmetry breaking,''
JHEP \textbf{08} (2005) 007,
\href{https://arxiv.org/abs/hep-th/0505076}{arXiv:hep-th/0505076}.

\bibitem{Chameleon}
J. Khoury, A. Weltman,
``Chameleon fields: Awaiting surprises for tests of gravity in space,''
Phys. Rev. Lett. \textbf{93} (2004) 171104,
\href{https://arxiv.org/abs/astro-ph/0309300}{arXiv:astro-ph/0309300}.

\bibitem{GKP}
S.B. Giddings, S. Kachru, J. Polchinski,
``Hierarchies from fluxes in string compactifications,''
Phys. Rev. D \textbf{66} (2002) 106006,
\href{https://arxiv.org/abs/hep-th/0105097}{arXiv:hep-th/0105097}.

\bibitem{Denef}
F. Denef, M.R. Douglas,
``Distributions of flux vacua,''
JHEP \textbf{05} (2004) 072,
\href{https://arxiv.org/abs/hep-th/0404116}{arXiv:hep-th/0404116}.

\bibitem{Cicoli}
M. Cicoli, J.P. Conlon, F. Quevedo,
``General analysis of LARGE Volume scenarios with string loop moduli stabilisation,''
JHEP \textbf{10} (2008) 105,
\href{https://arxiv.org/abs/0805.1029}{arXiv:0805.1029}.

\bibitem{Racetrack}
J.J. Blanco-Pillado et al.,
``Racetrack inflation,''
JHEP \textbf{11} (2004) 063,
\href{https://arxiv.org/abs/hep-th/0406230}{arXiv:hep-th/0406230}.

\bibitem{SPARC}
F. Lelli, S.S. McGaugh, J.M. Schombert,
``SPARC: Mass Models for 175 Disk Galaxies with Spitzer Photometry and Accurate Rotation Curves,''
AJ \textbf{152} (2016) 157,
\href{https://arxiv.org/abs/1606.09251}{arXiv:1606.09251}.

\bibitem{Paper4}
J.E. Slama,
``A Hidden Conservation Law of Gravity: Multi-Scale Validation of the QO+R Framework,''
Paper 4 in QO+R Framework Series (2025),
Repository: \href{https://github.com/JonathanSlama/QO-R-JEDSLAMA}{github.com/JonathanSlama/QO-R-JEDSLAMA}.

\end{thebibliography}

\end{document}
